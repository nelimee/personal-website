\documentclass[crop,tikz,convert={outext=.svg,command=\unexpanded{pdf2svg \infile\space\outfile}},multi=false]{standalone}
\newcommand{\ket}[1]{\ensuremath{\left\vert #1 \right\rangle}}
\newcommand{\bra}[1]{\ensuremath{\left\langle #1 \right\vert}}

\usepackage{../quantumcircuit}
\usetikzlibrary{backgrounds}

\begin{document}
\begin{tikzpicture}[background rectangle/.style={fill=white}, show background rectangle]
  \let\qmwires\empty
  % define initial positions of the quantum wires
  \xdef\dy{0.7}
  \defgateminsize[1.3em];
  \defwire (A) at ({-\dy});
  \defwire (B) at ({-2*\dy});
  % draw wires
  \xdef\dt{.7}
  \drawwires [\dt] (5);
  \drawlastmeasurewire [\dt] (A) (5) (1);
  \drawlastwire [\dt] (B) (5) (1);
  % Draw gates
  \gate (A-1) [$H$];
  \gatebackgroundcolor{blue}{
    \gate (A-2) [$S^\dagger$];
  }
  \ctrlgate (A-3) (B-3) [$U$];
  \gate (A-4) [$H$];
  \meas (A-5) [];
  % Draw outputs/inputs
  \node  at ($(B-0)!0.2!(B-1)$) {$/^n$};
  \inputlabel  (A-0)  [$\ket{ 0 }$];
  \inputlabel  (B-0)  [$\ket{ \psi }$];
  \outputlabel (A-6)  [$o_1$];
  \outputlabel (B-6)  [$\ket{\phi}$];
\end{tikzpicture}
\end{document}

% Local Variables:
% TeX-command-extra-options: "-shell-escape"
% End: