\documentclass[convert={density=1000,outext=.png}]{standalone}

\newcommand{\ket}[1]{\ensuremath{\left\vert #1 \right\rangle}}
\newcommand{\bra}[1]{\ensuremath{\left\langle #1 \right\vert}}

\usepackage{../quantumcircuit}
\usetikzlibrary{backgrounds}

\begin{document}
\begin{tikzpicture}[background rectangle/.style={fill=white}, show background rectangle]
  \let\qmwires\empty
  % define initial positions of the quantum wires
  \xdef\dy{0.7}
  \defgateminsize[1.3em];
  \defwire (A) at ({-\dy});
  \defwire (B) at ({-2*\dy});
  \defwire (C) at ({-3*\dy});
  % draw wires
  \xdef\dt{.7}
  \drawwires [\dt] (4);
  \drawlastmeasurewire[\dt] (A) (4) (1);
  \drawlastwire[\dt] (B) (4) (1);
  \drawlastwire[\dt] (C) (4) (1);
  
  % Draw gates
  \gate (A-1) [$H$];
  \ctrlswapgate (A-2) (B-2) (C-2);
  \gate (A-3) [$H$];
  \meas (A-4) [];
  % Draw outputs/inputs
  \node  at ($(B-0)!0.2!(B-1)$) {$/^n$};
  \node  at ($(C-0)!0.2!(C-1)$) {$/^n$};
  \inputlabel  (A-0)  [$\ket{ 0 }$];
  \inputlabel  (B-0)  [$\ket{ \psi_1 }$];
  \inputlabel  (C-0)  [$\ket{ \psi_2 }$];
  \outputlabel (A-5)  [$o_a$];
\end{tikzpicture}
\end{document}

% Local Variables:
% TeX-command-extra-options: "-shell-escape"
% End: