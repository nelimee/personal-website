\documentclass[convert={density=1000,outext=.png}]{standalone}

\newcommand{\ket}[1]{\ensuremath{\left\vert #1 \right\rangle}}
\newcommand{\bra}[1]{\ensuremath{\left\langle #1 \right\vert}}

\usepackage{../quantumcircuit}
\usetikzlibrary{backgrounds}

\begin{document}
\begin{tikzpicture}[background rectangle/.style={fill=white}, show background rectangle]
  \let\qmwires\empty
  % define initial positions of the quantum wires
  \xdef\dy{0.9}
  \defgateminsize[1.2em];
  \defwire (A) at ({-\dy});
  \defwire (B) at ({-2*\dy});
  \defwire (C) at ({-2.5*\dy});
  \defwire (D) at ({-3*\dy});
  \defwire (E) at ({-3.5*\dy});
  \defwire (F) at ({-4*\dy});
  \defwire (G) at ({-5*\dy});
  \defwire (H) at ({-5.5*\dy});
  \defwire (I) at ({-6*\dy});
  \defwire (J) at ({-6.5*\dy});
  \defwire (K) at ({-7*\dy});
   % draw wires
  \xdef\dt{1}
  \drawwires [\dt] (3);
  \drawwire [\dt] (A) (3) (4);
  \drawwire [\dt] (B) (3) (4);
  \drawwire [\dt] (F) (3) (4);
  \drawwire [\dt] (G) (3) (4);
  \drawwire [\dt] (K) (3) (4);
  \drawlastmeasurewire [\dt] (A) (7) (1);
  \drawlastmeasurewire [\dt] (B) (7) (1);
  \drawlastmeasurewire [\dt] (F) (7) (1);
  \drawlastmeasurewire [\dt] (G) (7) (1);
  \drawlastmeasurewire [\dt] (K) (7) (1);
  % Draw gates
  \gate (A-1) [$H$];
  \gate (B-1) [$R_y(\alpha_0)$];
  \gate (C-1) [$R_y(\alpha_1)$];
  \gate (D-1) [$R_y(\alpha_2)$];
  \gate (E-1) [$R_y(\alpha_3)$];
  \gate (F-1) [$R_y(\alpha_4)$];
  \gate (G-1) [$H$];
  \gate (I-1) [$H$];
  \gate (J-1) [$H$];
  \gate (K-1) [$H$];
  \ctrlgate (A-2) (G-2) [$X$];
  \ctrlgate (A-3) (H-3) [$Z$];
  \cnotgate (B-4) (G-4);
  \cnotgate (F-5) (K-5);
  \dotlink (B-4) (F-5);
  \dotlink (G-4) (K-5);
  \gatebackgroundcolor{blue}{%
    \gate (A-5) [$S^\dagger$];
  }
  \gate (A-6) [$H$];
  \gate (B-6) [$H$];
  \dotlink (B-6) (F-6);
  \gate (F-6) [$H$];
  \meas (A-7) [z];
  \meas (B-7) [z];
  \dotlink (B-7) (F-7);
  \meas (F-7) [z];
  \meas (G-7) [z];
  \dotlink (G-7) (K-7);
  \meas (K-7) [z];
  % Draw outputs/inputs
  \inputlabel  (A-0)  [$\ket{ 0 }$];
  \inputlabel  (B-0)  [$\ket{ 0 }$];
  \inputlabel  (C-0)  [$\ket{ 0 }$];
  \inputlabel  (D-0)  [$\ket{ 0 }$];
  \inputlabel  (E-0)  [$\ket{ 0 }$];
  \inputlabel  (F-0)  [$\ket{ 0 }$];
  \inputlabel  (G-0)  [$\ket{ 0 }$];
  \inputlabel  (H-0)  [$\ket{ 0 }$];
  \inputlabel  (I-0)  [$\ket{ 0 }$];
  \inputlabel  (J-0)  [$\ket{ 0 }$];
  \inputlabel  (K-0)  [$\ket{ 0 }$];
  \outputlabel (A-8)  [$o_a$];
  \outputlabel (B-8)  [$o_1^1$];
  \outputlabel (F-8)  [$o_n^1$];
  \outputlabel (G-8)  [$o_1^2$];
  \outputlabel (K-8)  [$o_n^2$];
  % Draw a red box around the destructive SWAP test
  \node [draw=red, fit=(B-4)(G-4)(K-5)(B-7)(K-7), outer sep=0pt, dashed, thick] (dSWAPtest) {};
  % Draw the remaining wires
  \draw let \p1=(C-3), \p2=(dSWAPtest.west) in (\x1, \y1) -- (\x2, \y1);
  \draw let \p1=(D-3), \p2=(dSWAPtest.west) in (\x1, \y1) -- (\x2, \y1);
  \draw let \p1=(E-3), \p2=(dSWAPtest.west) in (\x1, \y1) -- (\x2, \y1);
  \draw let \p1=(H-3.east), \p2=(dSWAPtest.west) in (\x1, \y1) -- (\x2, \y1);
  \draw let \p1=(I-3), \p2=(dSWAPtest.west) in (\x1, \y1) -- (\x2, \y1);
  \draw let \p1=(J-3), \p2=(dSWAPtest.west) in (\x1, \y1) -- (\x2, \y1);
\end{tikzpicture}
\end{document}

% Local Variables:
% TeX-command-extra-options: "-shell-escape"
% End: